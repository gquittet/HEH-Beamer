%!TEX root = ../../presentation.tex


\newcommand{\annee}[1]{
    \def\varAnnee{#1}
}


\newcommand{\configure}[2]{
    \def#1{#2}
}


\newcommand{\email}[1]{
    \def\varEmail{#1}
}


\newcommand{\hehBeamerInfo}{
    \begin{slide}
        \begin{center}
            Cette présentation a été réalisée avec le Framework HEH Beamer\\
            HEH Beamer \hehBeamerVersion\\
            \small
            \url{\hehBeamerURL}
            \normalsize
        \end{center}
    \end{slide}
}


\newcommand{\interligne}[1]{
    \def\varInterligne{#1}
}


\newcommand{\nom}[1]{
    \def\varNom{#1}
}


\newcommand{\pageDeTitre}{
    \begin{frame}
        \titlepage{}
    \end{frame}
}


\newcommand{\partie}[2]{
    \def\laPartieCourante{#1}
    \section{#1}
    \input{slides/#2}
}


\newcommand{\pdfEtNotes}{
	\usepackage{pgfpages}
	\setbeameroption{show notes on second screen}
}


\newcommand{\pdfEtNotesADroite}{
	\pdfEtNotes
	\setbeameroption{show notes on second screen=right}
}


\newcommand{\pdfEtNotesAGauche}{
	\pdfEtNotes
	\setbeameroption{show notes on second screen=left}
}


\newcommand{\prenom}[1]{
    \def\varPrenom{#1}
}


\newcommand{\sousPartie}[1]{
    \def\laSousPartieCourante{#1}
    \subsection{#1}
}


\newcommand{\sousSousPartie}[1]{
    \def\laSousSousPartieCourante{#1}
    \subsubsection{#1}
}


\newcommand{\tableDesMatieres}{
    \begin{slide}[Table des matières]
        \tableofcontents[currentsection,
        currentsubsection,
        hideothersubsections,
        sectionstyle=show/hide,
        subsectionstyle=show/shaded/hide]
    \end{slide}
}


\newcommand{\titre}[1]{
    \def\varTitre{#1}
}


\newcommand{\titreAbrege}[1]{
    \def\varTitreAbrege{#1}
}
